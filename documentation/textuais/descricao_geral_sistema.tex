\chapter{Descrição Geral do Sistema}
\addcontentsline{toc}{chapter}{Descrição Geral do Sistema}

% Este capítulo tem como objetivo descrever de forma geral o sistema, o escopo e as principais funções. A descrição geral do sistema deve abrange os itens a seguir.

O projeto visa, através da análise estatística de dados meteorológicos, auxiliar o estudo de viabilidade antes de se instalar painéis fotovoltaicos para coleta de energia solar.

Para isso, serão coletados dados através de sensores conectados a um microcontrolador arduino. Inicialmente, prevemos captar informações de umidade do ar, temperatura e incidência de radiação solar.

Dados esses, que serão enviados através de requisições HTTP para uma API, serão armazenadas em banco de dados e então, será feita uma análise estatística desses dados e então, exibidos para o usuário final, através de uma aplicação web.

\section{Descrição do Problema}

Neste item deve ser descrito o problema que será resolvido com o desenvolvimento do sistema. As questões a seguir devem ser respondidas.

Quem é afetado pelo sistema?
Qual é o impacto do sistema?
Qual seria uma boa solução para o problema?

\section{Principais Envolvidos e suas Características}

\subsection{Usuários do Sistema}

Neste item deve ser descrito para qual tipo de empresa se destina o sistema, os tipos de usuários que utilizarão o sistema.

Estas informações são importantes para a definição de usabilidade G do sistema.

\subsection{Desenvolvedores do Sistema}

Neste item deve ser descrito os tipos de pessoas envolvidas em todo o desenvolvimento do sistema direta ou indiretamente.

Estas informações são importantes para a distribuição de responsabilidades e pontos-focais de desenvolvimento.

\subsection{Regras de Negócio}

Neste item devem ser descritas as regras de negócio relevantes para o sistema, como por exemplo, restrições de negócio, restrições de desempenho, tolerância à falhas, volume de informação a ser armazenada, estimativa de crescimento de volume, ferramentas de apoio, etc.
