\chapter{Descrição e arquitetura do banco de dados}
\addcontentsline{toc}{chapter}{Descrição e arquitetura do banco de dados}

O banco de dados utilizado foi o MongoDB, um banco de dados não relacional, o desenho do banco de dados é controlado pela aplicação, que define quis campos devem ser indexados e como os dados devem se "relacionar", cada informação capturada fica armazenada dentro de uma coleção, utilizando o formato JSON.

\section{Motivação}

A escolha de um banco de dados não relacional, foi motivada pela facilidade ao se trabalhar com esquemas maleaveis, podendo ter suas informações mutadas, deixando a cargo da aplicação a tomada de decisão.

O banco MongoDB foi escolhido pela sua facilidade ao trabalhar com escrita de dados concorrente, pois, as informações são, em primeiro instante processadas e armazenadas em cache, e já são retornadas para a aplicação, somente após, o MongoDB se encarrega de realizar a inserção dos dados em disco.

Com todas as vantagens do banco de dados, a aplicação tira vantagem, se beneficiando no que toca performance, disponibilidade e mutabilidade das informações.