\chapter{Análise dos dados capturados}
\addcontentsline{toc}{chapter}{Análise dos dados capturados}

As informações que foram capturadas pelo microcontrolador, ao serem requisitadas pelo usuário, são disponibilizadas através de uma interface gráfica, utilizando análise de estatística e gráficos.

\section{Filtros}

Os filtros utilizados pelo usuário, são a estação meteorológica que ele deseja visualizar, a data inicial e a data final para busca dos dados e o intervalo para calculo das médias.

\subsection{Tratamento de dados}

Os dados foram exibidos da mesma forma que foram capturados, com exceção das informações de intensidade de luz, que foi invertida para melhor exibição no gráfico, e pela medida de nível de chuva, que além de invertida, foi transformada em porcentagem.

\section{Resultados}

\subsection{Médias}

A média aplicada em cima dos dados, é separada por um intervalo de tempo, como exemplo, caso o usuário informe que o intervalo requisitado é de uma hora, os dados são agrupados por cada hora e então, uma média é aplicada nesses dados, exibindo as médias em intervalos de horas no gráfico.

\subsection{Mínimas e Máximas}

É exibida também uma lista com as mínimas e máximas daquela estação meteorológica no intervalo de tempo informado.