\documentclass[
	% -- opções da classe memoir --
	12pt,				% tamanho da fonte
	%openright,			% capítulos começam em pág ímpar (insere página vazia caso preciso)
	oneside,			% para impressão em recto e verso. Oposto a oneside
	a4paper,			% tamanho do papel.
	% -- opções da classe abntex2 --
	%chapter=TITLE,		% títulos de capítulos convertidos em letras maiúsculas
	%section=TITLE,		% títulos de seções convertidos em letras maiúsculas
	%subsection=TITLE,	% títulos de subseções convertidos em letras maiúsculas
	%subsubsection=TITLE,% títulos de subsubseções convertidos em letras maiúsculas
	% -- opções do pacote babel --
	english,			% idioma adicional para hifenização
	french,				% idioma adicional para hifenização
	spanish,			% idioma adicional para hifenização
	brazil				% o último idioma é o principal do documento
]{abntex2}

% ---
% Pacotes básicos
% ---
\usepackage{lmodern}			% Usa a fonte Latin Modern
\usepackage[T1]{fontenc}		% Selecao de codigos de fonte.
\usepackage[utf8]{inputenc}		% Codificacao do documento (conversão automática dos acentos)
\usepackage{lastpage}			% Usado pela Ficha catalográfica
\usepackage{indentfirst}		% Indenta o primeiro parágrafo de cada seção.
\usepackage{color}				% Controle das cores
\usepackage{graphicx}			% Inclusão de gráficos
\usepackage{float}
\usepackage{microtype} 			% para melhorias de justificação
\usepackage{multirow}           % para multiplas linhas em tabelas
\usepackage{listings}
% ---
% ---
% Pacotes de citações
% ---
\usepackage[brazilian,hyperpageref]{backref}	 % Paginas com as citações na bibl
\usepackage[alf]{abntex2cite}	% Citações padrão ABNT

% ---
% CONFIGURAÇÕES DE PACOTES
% ---

% ---
% Configurações do pacote backref
% Usado sem a opção hyperpageref de backref
\renewcommand{\backrefpagesname}{Citado na(s) página(s):~}
% Texto padrão antes do número das páginas
\renewcommand{\backref}{}
% Define os textos da citação
\renewcommand*{\backrefalt}[4]{
	\ifcase #1 %
		Nenhuma citação no texto.%
	\or
		Citado na página #2.%
	\else
		Citado #1 vezes nas páginas #2.%
	\fi}%
% ---

% ---
% Informações de dados para CcBraaAPA e FOLHA DE ROSTO
% ---
\titulo{Solus, um software para monitoramento de painéis solares}
\autor{Angelo Rodrigo Ribeiro da Silva\thanks{angelorodriigo.rs@gmail.com}}
\local{Boituva}
\data{2018}
\orientador{Prof. Dr. Marcelo Figueiredo Polido.}
\instituicao{
  INSTITUTO FEDERAL DE EDUCAÇÃO, CIÊNCIA E TECNOLOGIA DE SÃO PAULO - IFSP
  \par
  Curso de Análise e Desenvolvimento de Sistemas}
\tipotrabalho{Trabalho de conclusão de curso}
% O preambulo deve conter o tipo do trabalho, o objetivo,
% o nome da instituição e a área de concentração
\preambulo{Documentação para o software desenvolvido como trabalho de conclusão de curso para análise e desenvolvimento de sistemas no Instituto Federal de Educação, Ciência e Tecnologia de São Paulo, câmpus de Boituva.}
% ---

% ---
% Configurações de aparência do PDF final

% alterando o aspecto da cor azul
\definecolor{blue}{RGB}{0, 0, 0}

% informações do PDF
\makeatletter
\hypersetup{
     	%pagebackref=true,
		pdftitle={\@title},
		pdfauthor={\@author},
    	pdfsubject={\imprimirpreambulo},
	    pdfcreator={LaTeX with abnTeX2 and Compiled with Pdflatex},
		pdfkeywords={painel solar}{ifsp}{arduino},
		colorlinks=true,       		% false: boxed links; true: colored links
    	linkcolor=blue,          	% color of internal links
    	citecolor=blue,        		% color of links to bibliography
    	filecolor=magenta,      		% color of file links
		urlcolor=blue,
		bookmarksdepth=4
}
\makeatother
% ---

% ---
% Espaçamentos entre linhas e parágrafos
% ---

% O tamanho do parágrafo é dado por:
\setlength{\parindent}{1.3cm}

% Controle do espaçamento entre um parágrafo e outro:
\setlength{\parskip}{0.2cm}  % tente também \onelineskip

% ---
% compila o indice
% ---
\makeindex
% ---

% ----
% Início do documento
% ----
\begin{document}

% Seleciona o idioma do documento (conforme pacotes do babel)
\selectlanguage{brazil}

% Retira espaço extra obsoleto entre as frases.
\frenchspacing

% --------------------------------
% ELEMENTOS PRÉ-TEXTUAIS
% --------------------------------
% ---
% Capa
% ---
\imprimircapa
% ---

% ---
% Folha de rosto
% (o * indica que haverá a ficha bibliográfica)
% ---
\imprimirfolhaderosto*
% ---

%% ---
% Inserir a ficha bibliografica
% ---

% Isto é um exemplo de Ficha Catalográfica, ou ``Dados internacionais de
% catalogação-na-publicação''. Você pode utilizar este modelo como referência.
% Porém, provavelmente a biblioteca da sua universidade lhe fornecerá um PDF
% com a ficha catalográfica definitiva após a defesa do trabalho. Quando estiver
% com o documento, salve-o como PDF no diretório do seu projeto e substitua todo
% o conteúdo de implementação deste arquivo pelo comando abaixo:
%
% \begin{fichacatalografica}
%     \includepdf{fig_ficha_catalografica.pdf}
% \end{fichacatalografica}

\begin{fichacatalografica}
	\sffamily
	\vspace*{\fill}					% Posição vertical
	\begin{center}					% Minipage Centralizado
	\fbox{\begin{minipage}[c][8cm]{13.5cm}		% Largura
	\small
	\imprimirautor
	%Sobrenome, Nome do autor

	\hspace{0.5cm} \imprimirtitulo  / \imprimirautor. --
	\imprimirlocal, \imprimirdata-

	\hspace{0.5cm} \pageref{LastPage} p. : il. (algumas color.) ; 30 cm.\\

	\hspace{0.5cm} \imprimirorientadorRotulo~\imprimirorientador\\

	\hspace{0.5cm}
	\parbox[t]{\textwidth}{\imprimirtipotrabalho~--~\imprimirinstituicao,
	\imprimirdata.}\\

	\hspace{0.5cm}
		1. nodejs.
		2. arduino.
		2. meteorologia.
		I. Dr. Marcelo Figueiredo Polido.
		II. Instituto Federal de Educação, Ciência e Tecnologia de São Paulo.
		III. Tecnologia em análise e desenvolvimento de sistemas.
		IV. Solus, um software para monitoramento de painéis solares
	\end{minipage}}
	\end{center}
\end{fichacatalografica}
% ---


%% ---
% Inserir errata
% ---
\begin{errata}

\vspace{\onelineskip}

LOREM, I. D. S. \textbf{Lorem ipsum dolor sit amet, consectetur adipisicing elit, sed do eiusmod tempor incididunt ut labore et dolore magna aliqua. Ut enim ad minim veniam, quis nostrud exercitation ullamco laboris nisi ut aliquip ex ea commodo consequat.}: Duis aute irure dolor in reprehenderit in voluptate velit esse cillum dolore eu fugiat nulla pariatur. Excepteur sint occaecat cupidatat non proident, sunt in culpa qui officia deserunt mollit anim id est laborum.

\begin{table}[htb]
\center
\footnotesize
\begin{tabular}{|p{1.4cm}|p{1cm}|p{3cm}|p{3cm}|}
  \hline
   \textbf{Folha} & \textbf{Linha}  & \textbf{Onde se lê}  & \textbf{Leia-se}  \\
    \hline
    1 & 10 & auto-conclavo & autoconclavo\\
   \hline
\end{tabular}
\end{table}

\end{errata}
% ---


%% ---
% Inserir folha de aprovação
% ---

% Isto é um exemplo de Folha de aprovação, elemento obrigatório da NBR
% 14724/2011 (seção 4.2.1.3). Você pode utilizar este modelo até a aprovação
% do trabalho. Após isso, substitua todo o conteúdo deste arquivo por uma
% imagem da página assinada pela banca com o comando abaixo:
%
% \includepdf{folhadeaprovacao_final.pdf}
%
\begin{folhadeaprovacao}

  \begin{center}
    {\ABNTEXchapterfont\large\imprimirautor}

    \vspace*{\fill}\vspace*{\fill}
    \begin{center}
      \ABNTEXchapterfont\bfseries\Large\imprimirtitulo
    \end{center}
    \vspace*{\fill}

    \hspace{.45\textwidth}
    \begin{minipage}{.5\textwidth}
        \imprimirpreambulo
    \end{minipage}%
    \vspace*{\fill}
  \end{center}

  \begin{center}
   \imprimirlocal, 4 de dezembro de 2018
  \end{center}

  Banca examinadora:

  \begin{center}
    \noindent\rule{16cm}{0.4pt}
    \vspace{0.1mm}
    (Titulação, nome completo, instituição)

    \vspace{5mm}

    \noindent\rule{16cm}{0.4pt}
    \vspace{0.1mm}
    (Titulação, nome completo, instituição)

    \vspace{5mm}

    \noindent\rule{16cm}{0.4pt}
    \vspace{0.1mm}
    (Titulação, nome completo, instituição)
  \end{center}

\end{folhadeaprovacao}
% ---


%\input{dedicatoria}

%% ---
% Agradecimentos
% ---
\begin{agradecimentos}
Agradecimentos

\end{agradecimentos}
% ---


%% ---
% Epígrafe
% ---
\begin{epigrafe}
    \vspace*{\fill}
\end{epigrafe}
% ---


%% ---
% RESUMOS
% ---

% resumo em português
\setlength{\absparsep}{18pt} % ajusta o espaçamento dos parágrafos do resumo
\begin{resumo}
  O trabalho desenvolvido se baseia na dificuldade durante a análise de dados meteorológicos na prévia instalação de painéis solares. Este documento visa descrever uma aplicação web focada em captar e descrever dados meteorológicos utilizando métricas de estatística.

 \textbf{Palavras-chave}: nodejs. painéis solares. energia solar. arduino.
\end{resumo}

% resumo em inglês
\begin{resumo}[Abstract]
 \begin{otherlanguage*}{english}
  The thesis developed is based on the difficulty in the analysis of meteorological data in the installation of solar panels. This document aims to describe a web software focused on capturing and describing meteorological data using statistics metrics.

   \vspace{\onelineskip}

   \noindent
   \textbf{Keywords}: nodejs. solar panels. solar energy. arduino.
 \end{otherlanguage*}
\end{resumo}


% ---
% inserir lista de ilustrações
% ---
\pdfbookmark[0]{\listfigurename}{lof}
\listoffigures*
\cleardoublepage
% ---

% ---
% inserir lista de tabelas
% ---
\pdfbookmark[0]{\listtablename}{lot}
\listoftables*
\cleardoublepage
% ---

% ---
% inserir lista de abreviaturas e siglas
% ---
\begin{siglas}
  \item[IFSP] Instituto Federal de Educação, Ciência e Tecnologia de São Paulo
\end{siglas}
% ---


%% ---
% inserir lista de símbolos
% ---
\begin{simbolos}

\end{simbolos}
% ---


% ---
% inserir o sumario
% ---
\pdfbookmark[0]{\contentsname}{toc}
\tableofcontents*
\cleardoublepage
% ---


% --------------------------------
% ELEMENTOS TEXTUAIS
% --------------------------------
\textual


% ----------------------------------------------------------
% Introdução (exemplo de capítulo sem numeração, mas presente no Sumário)
% ----------------------------------------------------------
\chapter{Introdução}
\addcontentsline{toc}{chapter}{Introdução}
% ----------------------------------------------------------

Na sociedade contemporânea, diversas preocupações quanto a captação de energia surgiram. Uma dessas preocupações é cada vez mais, buscar fontes renováveis de energia.

Atualmente, a energia solar vem mostrando seus benefícios, sendo pelo custo, que é muito mais baixo do que pás eólicas e pela facilidade de instalação, que pode ser feita sem a necessidade de uma grande área reservada.

Devido ao avanço da captação de energia solar, diversos desafios surgiram ao se estudar a melhor forma de se trabalhar com a energia captada.

\section{Tema}

Captação de dados meteorológicos por sensores e microcontroladores arduino, para que se faça a análise estatística de dados.

\section{Objetivo do Projeto}

Conseguir a melhor obtenção e utilização de energia solar de painéis fotovoltaicos através de captação e análise prévia dos dados meteorológicos, dados esses que precisam ser disponibilizados da maneira mais fácil possível.

\section{Delimitação do Problema}

Não existe uma forma prática de realizar a análise dos dados antes da instalação de painéis solares, visto que, os dados captados por sensores, para que seja feita a análise, possui um fluxo muito alto de informações, assim, a necessidade de uma aplicação que faça a análise dessa quantidade massiva de dados, se faz evidente.

\section{Justificativa da Escolha do Tema}

Existe um projeto de instalação de uma usina solar no IFSP, no campus localizado em Boituva, portanto, o tema do projeto foi escolhido, para que se possa, no futuro, trabalhar a energia captada por painéis solares da melhor forma possível.

\section{Método de Trabalho}

A metodologia de trabalho escolhida para este projeto, utiliza a metodologia de desenvolvimento de software SCRUM com entregas incrementais, para a implementação do projeto, foi decidido a utilização de painéis microcontroladores arduino, enviando requisições HTTP para uma api, construída em PHP e utilizando banco de dados SQL.

\section{Organização do Trabalho}

Neste item deve-se descrever como o documento está organizado.



\chapter{Descrição Geral do Sistema}
\addcontentsline{toc}{chapter}{Descrição Geral do Sistema}

% Este capítulo tem como objetivo descrever de forma geral o sistema, o escopo e as principais funções. A descrição geral do sistema deve abrange os itens a seguir.

O projeto visa, através da análise estatística de dados meteorológicos, auxiliar o estudo de viabilidade acerca da instalação de painéis fotovoltaicos.

Para isso, serão coletados dados através de sensores conectados a um microcontrolador arduino. Inicialmente, prevemos captar informações de umidade do ar, temperatura e incidência de radiação solar.

Dados esses, que serão enviados através de requisições HTTP para uma API, serão armazenadas em banco de dados e então, será feita uma análise estatística desses dados.

A interface do usuário final com a aplicação, será feita através de uma aplicação web, onde os dados analisados serão disponibilizados, onde o usuário fará consultas a essas informações.

\section{Descrição do Problema}

Durante o estudo de viabilidade acerca da instalação de painéis fotovoltaicos no IFSP, notou-se uma dificuldade na captura e análise dos dados para tomada de decisão, justificando assim, a necessidade da automatização desse processo, considerando também, a quantidade massiva de dados e as possíveis falhas de estimativa pelo cálculo humano.

Pelo ainda alto de custo de instalação de painéis solares, uma decisão errada na instalação de painéis solares poderia causar um dano financeiro imensurável.

O sistema afeta principalmente, a configuração dos painéis como ângulo, posição, local, entre outras variaveis que poderiam, através da análise de dados serem melhor decididas.

\section{Principais Envolvidos e suas Características}

\subsection{Usuários do Sistema}

Neste item deve ser descrito para qual tipo de empresa se destina o sistema, os tipos de usuários que utilizarão o sistema.

Estas informações são importantes para a definição de usabilidade G do sistema.

\subsection{Desenvolvedores do Sistema}

Neste item deve ser descrito os tipos de pessoas envolvidas em todo o desenvolvimento do sistema direta ou indiretamente.

Estas informações são importantes para a distribuição de responsabilidades e pontos-focais de desenvolvimento.

\subsection{Regras de Negócio}

Neste item devem ser descritas as regras de negócio relevantes para o sistema, como por exemplo, restrições de negócio, restrições de desempenho, tolerância à falhas, volume de informação a ser armazenada, estimativa de crescimento de volume, ferramentas de apoio, etc.


% ----------------------------------------------------------
% Finaliza a parte no bookmark do PDF
% para que se inicie o bookmark na raiz
% e adiciona espaço de parte no Sumário
% ----------------------------------------------------------
\phantompart

\input{textuais/conclusao}


% --------------------------------
% ELEMENTOS PÓS-TEXTUAIS
% --------------------------------
% ----------------------------------------------------------
% ELEMENTOS PÓS-TEXTUAIS
% ----------------------------------------------------------
\postextual
% ----------------------------------------------------------

% ----------------------------------------------------------
% Referências bibliográficas
% ----------------------------------------------------------
\bibliography{references}

% ----------------------------------------------------------
% Glossário
% ----------------------------------------------------------
%
% Consulte o manual da classe abntex2 para orientações sobre o glossário.
%
%\glossary

%\input{pos-textuais/apendices}

%% ----------------------------------------------------------
% Anexos
% ----------------------------------------------------------

% ---
% Inicia os anexos
% ---
\begin{anexosenv}

% Imprime uma página indicando o início dos anexos
\partanexos

% ---
\chapter{Morbi ultrices rutrum lorem.}
% ---

Lorem ipsum dolor sit amet, consectetur adipisicing elit, sed do eiusmod tempor incididunt ut labore et dolore magna aliqua. Ut enim ad minim veniam, quis nostrud exercitation ullamco laboris nisi ut aliquip ex ea commodo consequat. Duis aute irure dolor in reprehenderit in voluptate velit esse cillum dolore eu fugiat nulla pariatur. Excepteur sint occaecat cupidatat non proident, sunt in culpa qui officia deserunt mollit anim id est laborum.

% ---
\chapter{Cras non urna sed feugiat cum sociis natoque penatibus et magnis dis
parturient montes nascetur ridiculus mus}
% ---

Lorem ipsum dolor sit amet, consectetur adipisicing elit, sed do eiusmod tempor incididunt ut labore et dolore magna aliqua. Ut enim ad minim veniam, quis nostrud exercitation ullamco laboris nisi ut aliquip ex ea commodo consequat. Duis aute irure dolor in reprehenderit in voluptate velit esse cillum dolore eu fugiat nulla pariatur. Excepteur sint occaecat cupidatat non proident, sunt in culpa qui officia deserunt mollit anim id est laborum.

\end{anexosenv}


%---------------------------------------------------------------------
% INDICE REMISSIVO
%---------------------------------------------------------------------
\phantompart
\printindex
%---------------------------------------------------------------------


\end{document}
