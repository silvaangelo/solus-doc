\chapter{Descrição e documentação da API}
\addcontentsline{toc}{chapter}{Descrição e documentação da API}

Podemos conferir abaixo a documentação das rotas da API.

\begin{table}[H]
    \centering
    \caption{Descrição das rotas da API}
    \label{table_api_routes}
    \begin{tabular}{|l|l|l|}
    \hline
    \textbf{Método}  & \textbf{Rota}        & \textbf{Descrição}                           \\ \hline
    GET              & /arduino             & Lista os arduinos                            \\ \hline
    GET              & /arduino/:id         & Retorna os dados de um arduino               \\ \hline
    POST             & /arduino             & Cadastra um novo arduino                     \\ \hline
    POST, PATCH, PUT & /arduino/:id         & Atualiza os dados de um arduino              \\ \hline
    DELETE           & /arduino/:id         & Deleta um arduino e suas medidas capturadas  \\ \hline
    GET              & /user                & Lista os usuários                            \\ \hline
    GET              & /user/:id            & Retorna os dados de um usuário               \\ \hline
    POST             & /user                & Cadastra um novo usuário                     \\ \hline
    POST, PATCH, PUT & /user/:id            & Atualiza os dados de um usuário              \\ \hline
    DELETE           & /user/:id            & Deleta um usuário                            \\ \hline
    POST             & /user/login          & Recebe as credenciais e retorna o token      \\ \hline
    POST             & /measure             & Cadastra uma medida capturada                \\ \hline
    GET              & /statistic/:id       & Retorna as estatísticas de dados do arduino  \\ \hline
    \end{tabular}
\end{table}

\section{Métodos HTTP utilizados}

Como podemos visualizar na tabela \ref{table_api_routes}, o método POST fica designado a criar um recurso quando a requisição for enviada para uma rota sem a identificação. O método POST também é utilizado para atualizar um recurso quando a requisição for enviada para um recurso identificado. Para a atualização de um recurso também pode ser utilizado o método PATCH.
Para a consulta de recursos é utilizado o método GET, o método DELETE é utilizado para excluir um recurso.

\section{Autenticação}

A autenticação foi feita utilizando Json Web Token, onde, uma vez que o usuário tenha feito o login na API, as informações do usuário e seu token de acesso são retornados.

O token é passado para as rotas através dos cabeçalhos da requisição, onde a autenticação se faz necessária, o token enviado é validado, e caso seja invalido, o acesso ao usuário é negado.